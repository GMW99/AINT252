\section{Formal Languages and Automata}
Pattern matching is central to many applications and supported by regular expressions in programming languages. The Chomsky hierarchy of Formal Languages provides a theory of programming languages of increasing expressively. Formal Automata or Machines are models of computation, that implement languages of the Chomsky hierarchy.
\subsection{Regular Expressions}
A regular expression describes a pattern, it has very basic operators however with these operations it can specify a usually infinite set of strings see table \ref{tab:regex}. \footnote{RegExp cheat sheet \href{https://www.cheatography.com/davechild/cheat-sheets/regular-expressions/}{(link)}}
\begin{table}[h]
\begin{tabular}{llll}
\hline
\multicolumn{1}{|l|}{\textbf{Operation}} & \multicolumn{1}{l|}{\textbf{Regular Expression}} & \multicolumn{1}{l|}{\textbf{Yes}} & \multicolumn{1}{l|}{\textbf{No}} \\ \hline
\multicolumn{1}{|l|}{Concatenation} & \multicolumn{1}{l|}{aabbaab} & \multicolumn{1}{l|}{aabaab} & \multicolumn{1}{l|}{Every Other String} \\ \hline
\multicolumn{1}{|l|}{Wildcard} & \multicolumn{1}{l|}{.u.u.u.} & \multicolumn{1}{l|}{\begin{tabular}[c]{@{}l@{}}cumulus\\ jugulum\end{tabular}} & \multicolumn{1}{l|}{\begin{tabular}[c]{@{}l@{}}succubus\\ tumultuous\end{tabular}} \\ \hline
\multicolumn{1}{|l|}{Union} & \multicolumn{1}{l|}{aa | baab} & \multicolumn{1}{l|}{\begin{tabular}[c]{@{}l@{}}aa\\ babb\end{tabular}} & \multicolumn{1}{l|}{Every Other String} \\ \hline
\multicolumn{1}{|l|}{Closure} & \multicolumn{1}{l|}{ab*a} & \multicolumn{1}{l|}{\begin{tabular}[c]{@{}l@{}}aa\\ abbba\end{tabular}} & \multicolumn{1}{l|}{\begin{tabular}[c]{@{}l@{}}ab\\ ababa\end{tabular}} \\ \hline
\multicolumn{1}{|l|}{\multirow{2}{*}{Parentheses}} & \multicolumn{1}{l|}{a(a|b)aab} & \multicolumn{1}{l|}{\begin{tabular}[c]{@{}l@{}}aaaab\\ abaab\end{tabular}} & \multicolumn{1}{l|}{Every Other String} \\ \cline{2-4} 
\multicolumn{1}{|l|}{} & \multicolumn{1}{l|}{(ab)*a} & \multicolumn{1}{l|}{\begin{tabular}[c]{@{}l@{}}a\\ ababababa\end{tabular}} & \multicolumn{1}{l|}{\begin{tabular}[c]{@{}l@{}}\varepsilon\\ abbbaa\end{tabular}} \\ \hline
\end{tabular}
\caption{Regex}
\label{tab:regex}
\end{table}
\subsubsection{Generalised Regular Expressions}
These have all of the rules previously mentioned however they also have more as seen in the table \ref{tab:Gregex}
\begin{table}[H]
\begin{tabular}{llll}
\hline
\multicolumn{1}{|l|}{\textbf{Operation}} & \multicolumn{1}{l|}{\textbf{Regular Expression}} & \multicolumn{1}{l|}{\textbf{Yes}} & \multicolumn{1}{l|}{\textbf{No}} \\ \hline
\multicolumn{1}{|l|}{One or more} & \multicolumn{1}{l|}{a(bc)+de} & \multicolumn{1}{l|}{\begin{tabular}[c]{@{}l@{}}abcde\\ abcbcde\end{tabular}} & \multicolumn{1}{l|}{\begin{tabular}[c]{@{}l@{}}ade\\ bcde\end{tabular}} \\ \hline
\multicolumn{1}{|l|}{Character classes} & \multicolumn{1}{l|}{{[}A-Za-z{]}{[}a-z{]}*} & \multicolumn{1}{l|}{\begin{tabular}[c]{@{}l@{}}capitalised\\ Word\end{tabular}} & \multicolumn{1}{l|}{\begin{tabular}[c]{@{}l@{}}camelCase\\ 4illegal\end{tabular}} \\ \hline
\multicolumn{1}{|l|}{Exactly k} & \multicolumn{1}{l|}{{[}0-9{]}\{5\}-{[}0-9{]}\{4\}} & \multicolumn{1}{l|}{\begin{tabular}[c]{@{}l@{}}08540-1321\\ 19072-5541\end{tabular}} & \multicolumn{1}{l|}{\begin{tabular}[c]{@{}l@{}}111111111111\\ 166-54-111\end{tabular}} \\ \hline
\multicolumn{1}{|l|}{Negations} & \multicolumn{1}{l|}{{[}\textasciicircum{}aeiob{]}\{6\}} & \multicolumn{1}{l|}{rhythm} & \multicolumn{1}{l|}{decade} \\ \hline
\multirow{2}{*}{} &  &  &  \\
\end{tabular}
\caption{Generalised Regex}
\label{tab:Gregex}
\end{table}
\subsection{Deterministic Finite State Automata (DFA)}
A DFA $M$ is  a 5-tuple $M= \{Q,\Sigma,q_0,F,\delta\}$ where 
\begin{itemize}
    \item $Q$ is a finite set of states
    \item $\Sigma$ is finite set of input symbols ("alphabet")
    \item $q_0$ is a start state from $Q$
    \item F is a set of accepting states from $Q$
    \item $\delta$ is a transition function, i.e. a total mapping from $Q\times\Sigma$ to $Q$
\end{itemize}
\subsubsection{Transition Function}
The transition function $\delta$ takes a state $q$ and an input symbol $a$ and maps them to a state $q'$. The function is 'total' for each pair $(q,a)$ exactly one target state $q$ must exist so $\delta(q,a)=q'$ this function is often written as $q \xrightarrow{a}q'$. $\delta$ can be represented by a matrix $T$ in code.
\subsubsection{Example of what the symbols mean}
Using the figure \ref{fig:DFAB*} and calling it $M$ so $M= \{Q,\Sigma,q_0,F,\delta\}$ the symobls represent
\begin{itemize}
    \item $Q$ is $q_0,q_1$
    \item $\Sigma$ is a,b
    \item $q_0$ is $q_0$
    \item F is $q_0$
    \item $\delta(q_0,a) = q_1$, $\delta(q_0,b) = q_0$, $\delta(q_1,a) = q_1$, $\delta(q_1,b) = q_1$
\end{itemize}
\begin{figure}[H]
    \centering
  \begin{tikzpicture}[shorten >=1pt,node distance=2cm,on grid,auto] 
   \node[state,initial,accepting] (q_0)   {$q_0$}; 
   \node[state] (q_1) [right=of q_0] {$q_1$}; 
    \path[->] 
    (q_0) edge  node {a} (q_1)
          edge [loop above] node {b} ()
    (q_1) edge  [loop above] node {a,b} ();
  \end{tikzpicture} 
    \caption{DFA of regex b*}
    \label{fig:DFAB*}
\end{figure}
\subsubsection{How a DFA Works}
A run of a DFA $M$ on $s$ = $a_0,a_1,..,a_{n-1}$ is a sequence of states $q_0,q_1,q_2,..,q_n$ such that $q_i \xrightarrow{a}q_{i+1}\,\, \forall \,\, 0\leq i \le n$. The determinism part means for a given input word a DFA has a unique run, because the transition function is a total function. A DFA accepts a word if $q_n$ is in the set of final state $F$; otherwise it rejects the word.
\subsubsection{Informal definition}
In figure \ref{fig:DFSA} the DFA has three states $q_0$,$q_1$ and $q_2$ with $q_1$ being the goal state and $q_0$ being the start state.
\begin{figure}[H]
    \centering
  \begin{tikzpicture}[shorten >=1pt,node distance=2cm,on grid,auto] 
   \node[state,initial] (q_0)   {$q_0$}; 
   \node[state,accepting] (q_1) [above right=of q_0] {$q_1$}; 
   \node[state] (q_2) [below right=of q_0] {$q_2$}; 
    \path[->] 
    (q_0) edge  node {a} (q_1)
          edge  node [swap] {b} (q_2)
    (q_1) edge [loop above] node {a,b} ()
    (q_2) edge [loop below] node {a,b} ();
\end{tikzpicture} 
    \caption{DFA}
    \label{fig:DFSA}
\end{figure}
The DFA in figure \ref{fig:DFSA} reads inputs symbols as a's or b's and transits between states as indicated by the labelled arrows. 
\\\\
In a DFA there can only be one starting state however there can be multiple goal states which is represented as by the double circle. Once an all inputs have been read if the end state is a goal state then the string is correct/accepted otherwise it is wrong/rejected. 
\subsubsection{Pseudo code}
\begin{algorithm}[H]
\SetAlgoLined
\KwResult{True or False }
 T = [1,0;1,1]\;
 
 q = $q_0$\;
 
 a = $a_0,a_1,..,a_{n-1}$\;
 
 \\
 \While{$i$ $\le$ $N-1$}{
    q = T[q,a[i]]\;
 }
\eIf{q is in $F$}{
return True\;
}{
return False\;
}
 \caption{DFA Algorithms}
\end{algorithm}
\subsection{Non-Deterministic Finite State Automata (NFA)}
In a DFA, each pair of state and input symbol uniquely defines a next state, however, in Non-determinism allows for missing and non-unique transitions (and possibly for more than 1 start state). If multiple targets exist (non-determinism!) the machine is “cloned” and all alternatives run further. There may be “epsilon-transitions” (without an input). If a target state cannot be determined the machine dies
A word is accepted if at least one derivation exists, i.e. if at least
one machine survives in an accepting state.
\subsubsection{NFA example}
The NFA figure \ref{fig:NFA*} can have multiple runs with the same word for example $abbb$ can end up in either $s_1$ or $s_2$. An NFA can have 0,1 or more than one runs on a given word.
\begin{figure}[H]
    \centering
  \begin{tikzpicture}[shorten >=1pt,node distance=2cm,on grid,auto] 
   \node[state,initial] (s_1)   {$s_1$}; 
   \node[state,accepting] (s_2) [right=of s_1] {$s_2$}; 
    \path[->] 
    (s_1) edge  node {b} (s_2)
          edge [loop above] node {a,b} ()
    (s_2) edge  [loop above] node {b} ();
  \end{tikzpicture} 
    \caption{NFA of words ending in b}
    \label{fig:NFA*}
\end{figure}
\subsection{Formal Languages}
Finite state machines accept or reject words, the set of all words an FSA called A accepts is called the Language it accepts 
\begin{equation}
    L(A) = \{u \in \Sigma^* \,\,|\,\, A \text{ has an accepting run on }u\}
\end{equation}
Regular Expressions characterise words as well, thereby they also define Languages.
\subsubsection{Equivalence (without proofs)}
For each DFA exists a regular expression that defines the same Language as the DFA.For each regular expression exists a DFA that accepts the same language as the RegExp For each NFA exists a DFA that accepts the same language (i.e. NFAs are not stronger than DFAs) DFAs, NFAs, and regular expressions all compute the same class of languages. These are the regular languages
\subsection{Grammars and Languages}

