\section{Introduction Lecture}
This lecture covered a quick introduction to the module about AI ending with a formal definition of a simple neural network and types of learning. This lecture was presented by Prof Roman Borisyuk and some of the information is directly copied from the lecture slides.
\subsection{Introduction to AI}
These days, AI is becoming a more and more popular approach for solving many different real-life problems. Since the time of the first computers (1950$^{th}$), AI has been an active scientific discipline.
\\\\
The field of AI draws upon many different interdisciplinary fields such as; computer science, mathematics, information theory, engineering, psychology, cognitive science and many more.
\\\\
AI is a challenging goal in which we want computer code (an agent) to mimic/ have natural intelligence, i.e. a human brain. Alan Turing in his paper "Computing Machinery and Intelligence. Mind 49:433-460 (1050) formulated a question: Can a machine think?. Turing suggested a way to test this would be to use the "imitation game."
\footnote{If you do not want to read the whole paper this overview is very detailed and give the gist of the paper (\href{https://blog.acolyer.org/2017/10/20/computing-machinery-and-intelligence/}{link})}
\subsection{Natural AI}
A simple but attractive idea for designing AI is to look at the human brain as this is the most complex and only thing we know of as General Intelligence which is ultimately our goal with AI, because of this a lot of our models are based on our understanding of how the brain works, i.e. the Biological Neuron.
\subsubsection{Biological Neuron}
A neuron consists of the cell body (soma), input (dendrite) and an output (axon). To communicate between neurons is via the synapse, and the synaptic weight defines the efficacy of this transmission \footnote{This is also known as the connection strength and is adjustable}. A neuronal network is a set of interconnected neurons, and a neuron integrates all the inputs signals and compares the inputs with the threshold: if it exceeds the threshold, then the neuron generates an action potential.
\\\\
We try and mimic this structure in neural networks as you will see in the next section.
\newpage
\subsection{Formal Neural Network}
A basic Neural Network can be written like so, and an example of what this network would look like is shown in figure \ref{fig:NeuralNetwork:4-5-2} \footnote{There is a series of lectures on youtube from 3Blue1Brown covering this topic in great detail (\href{https://www.youtube.com/watch?v=aircAruvnKk&list=PLZHQObOWTQDNU6R1_67000Dx_ZCJB-3pi}{link})}
\begin{itemize}
    \item Inputs: $x_1, x_2,..,x_n$; binary (0/1)
    \item Connection Strengths: $w_1,w_2,...,w_n$ 
    \item Summation $ h = \sum_{i=1}^{n} w_i x_i$
    \item Output $y = f(h)$; binary: if $h > T$ then $y = 0$ else $y =1$
\end{itemize}
Where $f(h)$ is the activation function with a threshold of $T$
\begin{figure}[H]
    \centering
  \begin{tikzpicture}[shorten >=1pt,->,draw=black!50, node distance=\layersep]
    \tikzstyle{every pin edge}=[<-,shorten <=1pt]
    \tikzstyle{neuron}=[circle,fill=black!25,minimum size=17pt,inner sep=0pt]
    \tikzstyle{input neuron}=[neuron, fill=green!50];
    \tikzstyle{output neuron}=[neuron, fill=red!50];
    \tikzstyle{hidden neuron}=[neuron, fill=blue!50];
    \tikzstyle{annot} = [text width=4em, text centered]

    % Draw the input layer nodes
    \foreach \name / \y in {1,...,4}
    % This is the same as writing \foreach \name / \y in {1/1,2/2,3/3,4/4}
        \node[input neuron, pin=left:Input \#\y] (I-\name) at (0,-\y) {};

    % Draw the hidden layer nodes
    \foreach \name / \y in {1,...,5}
        \path[yshift=0.5cm]
            node[hidden neuron] (H-\name) at (\layersep,-\y cm) {};

    % Draw the output layer node
    \node[output neuron,pin={[pin edge={->}]right:Output}, right of=H-2] (O) {};
    \node[output neuron,pin={[pin edge={->}]right:Output}, right of=H-4] (A) {};
    % Connect every node in the input layer with every node in the
    % hidden layer.
    \foreach \source in {1,...,4}
        \foreach \dest in {1,...,5}
            \path (I-\source) edge (H-\dest);

    % Connect every node in the hidden layer with the output layer
    \foreach \source in {1,...,5}
        \path (H-\source) edge (O);
    \foreach \source in {1,...,5}
        \path (H-\source) edge (A);

    % Annotate the layers
    \node[annot,above of=H-1, node distance=1cm] (hl) {Hidden layer};
    \node[annot,left of=hl] {Input layer};
    \node[annot,right of=hl] {Output layer};
\end{tikzpicture}  

    \caption{Neural network:4-5-2}
    \label{fig:NeuralNetwork:4-5-2}
\end{figure}
Neural Network is a set of coupled neurons Similar to real neurons; formal neurons are interconnected: the output of one neuron is an input to another neuron. The figure above demonstrates this.
\subsection{Learning}
Similar to a real neural network, the Artificial Neural Network (ANN) can learn by adjusting the connection strength. The current most popular types of learning in AI: supervised, unsupervised and reinforcement learning 
\\\\
For example, Deep Learning (Deep Neural Network) is a method to learn from the data; Deep Learning can use supervised learning: for a particular input vector, a "teacher" defines a desirable output. Thus, the deep neural network can be trained to find a correspondence between inputs and desirable outputs. 
 

